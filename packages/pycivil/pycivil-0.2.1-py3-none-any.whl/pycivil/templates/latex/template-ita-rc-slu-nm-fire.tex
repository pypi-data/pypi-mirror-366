Nella seguente figura sono rappresentati i due domini della sezione a caldo e a freddo. Il dominio a caldo è rappresentato in rosso.

%- if domain
\begin{figure}[!h]
    \centering
    \setlength{\fboxsep}{0pt}%
    \setlength{\fboxrule}{1pt}%
    \fbox{\includegraphics{\VAR{domainUrl}}}%
    \caption{Dominio di interazione SLU per pressoflessione retta ed incendio}
\end{figure}
%- endif

La sezione ridotta utilizzando il \textit{metodo della isoterma 500} ha forma \textit{rettangolare} con le seguenti caratteristiche geometriche sono:

\begin{align*}
b_{red} &= \VAR{ width } \text{ \textit{mm}} & h_{red} &= \VAR{ height } \text{ \textit{mm}}
\end{align*}

Per la mappatura termica della sezione sono stati utilizzati i seguenti parametri di incendio:

\begin{align*}
\text{Curva di incendio} &= \text{\VAR{ curve }} & \text{Tempo di incendio} &= \text{\VAR{ time }} \text{ \textit{min}}
\end{align*}

\begin{longtable}[c]{|c|>{\raggedleft\arraybackslash}p{20mm}|>{\raggedleft\arraybackslash}p{20mm}|>{\raggedleft\arraybackslash}p{20mm}|>{\raggedleft\arraybackslash}p{20mm}|>{\raggedleft\arraybackslash}p{20mm}|>{\centering\arraybackslash}p{15mm}|}
\caption{Verifiche SLU per pressoflessione retta e incendio\label{tab:SLU_NM_FIRE}} \\
\hline
\multirow{2}{*}{\textbf{id}} & \mcsym{N_{ed}}          & \multicolumn{1}{c|}{$\boldsymbol{M_{ed}}$} & \multicolumn{1}{c|}{$\boldsymbol{N_{er}}$} & \multicolumn{1}{c|}{$\boldsymbol{M_{ed}}$} & \multicolumn{1}{c|}{$\boldsymbol{FS}$} & $\boldsymbol{check}$ \bigstrut \\ \cline{2-7}
                             & \multicolumn{1}{c|}{\footnotesize{\textit{[KN]}}}     & \multicolumn{1}{c|}{\footnotesize{\textit{[KNm]}}} & \multicolumn{1}{c|}{\footnotesize{\textit{[KN]}}} & \multicolumn{1}{c|}{\footnotesize{\textit{[KNm]}}} & \multicolumn{1}{c|}{\footnotesize{\textit{[\ldots]}}} & \footnotesize{\textit{[\ldots]}}\\
\endfirsthead

\multicolumn{5}{c}{continuazione Tabella \ref{tab:SLU_NM_FIRE}}\\
\hline
\multirow{2}{*}{\textbf{id}} & \multicolumn{1}{c|}{$\boldsymbol{N_{ed}}$}            & \multicolumn{1}{c|}{$\boldsymbol{M_{ed}}$} & \multicolumn{1}{c|}{$\boldsymbol{N_{er}}$} & \multicolumn{1}{c|}{$\boldsymbol{M_{ed}}$} & \multicolumn{1}{c|}{$\boldsymbol{FS}$} & $\boldsymbol{check}$ \bigstrut \\ \cline{2-7}
                             & \multicolumn{1}{c|}{\footnotesize{\textit{[KN]}}}     & \multicolumn{1}{c|}{\footnotesize{\textit{[KNm]}}} & \multicolumn{1}{c|}{\footnotesize{\textit{[KN]}}} & \multicolumn{1}{c|}{\footnotesize{\textit{[KNm]}}} & \multicolumn{1}{c|}{\footnotesize{\textit{[\ldots]}}} & \footnotesize{\textit{[\ldots]}}\\
\endhead

\hline
%- for c in check_SLU_MN
\VAR{c.id} & \VAR{c.Ned} & \VAR{c.Med} & \VAR{c.Ner} & \VAR{c.Mer} & \VAR{c.FS} & \VAR{c.check} \\ \hline
%- endfor
\end{longtable}
