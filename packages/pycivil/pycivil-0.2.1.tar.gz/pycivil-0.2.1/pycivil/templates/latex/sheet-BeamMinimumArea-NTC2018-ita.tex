\centering

\begin{figure}[h!]
	\centering
	\begin{tabular}{ll}\toprule
		\multicolumn{2}{c}{\textbf{Dati generali}}\\\midrule
		Elemento strutturale: & \textbf{ \VAR{elementDescr} }  \\
		Norma utilizzata per il materiale: & \textbf{ \VAR{keyCode} }  \\
		\bottomrule
	\end{tabular}
\end{figure}

\begin{figure}[h!]
	\centering
	\begin{tabular}{lll}\toprule
		\multicolumn{3}{c}{\textbf{Caratteristiche dei materiali}}\\ \midrule
		Descrizione & Valore   & \mbox{u.d.m.} \\ \midrule
		Classe del calcestruzzo: & \textbf{ \VAR{concreteClass} } & $\ldots$ \\
		Resistenza cilindrica a compressione: & $f_{ck}=\VAR{cls_fck}$  & $N/mm^2$\\
		Resistenza media a trazione: & $f_{ctm}=\VAR{cls_fctm}$  & $N/mm^2$\\
		Classe di acciaio: &  \textbf{ \VAR{steelClass} }  & $\ldots$\\
		Resistenza caratteristica a snervamento: & $f_{yk}=\VAR{steel_fyk}$ & $N/mm^2$\\
		\bottomrule
	\end{tabular}
\end{figure}

\begin{figure}[h!]
	\centering
	\begin{tabular}{lll}\toprule
		\multicolumn{3}{c}{\textbf{Dati geometrici della sezione}}\\ \midrule
		Descrizione & Valore   & \mbox{u.d.m.} \\ \midrule
		Altezza dell'elemento: &  $H=\VAR{hEl}$  & $mm$ \\
		Larghezza dell'elemento: &  $W=\VAR{wEl}$  & $mm$ \\
		Larghezza media dell'elemento in zona tesa: &  $b_{t,med}=\VAR{bt}$  & $mm$ \\
		Diametro barre longitudinali: & $\phi_{lt} = \VAR{rebarD}$  &  $mm$ \\
		Copriferro barre longitudinali: &  $c = \VAR{cover}$  & $mm$ \\
		Larghezza minima dell'elemento a taglio: &  $b_{t,min}=\VAR{bMin}$  & $mm$ \\ \midrule
		Diametro barre longitudinali compresse: & $\phi_{lc} = \VAR{rebarDComp}$  &  $mm$ \\
		Diametro staffe o spilli: &  $\phi_{st}=\VAR{stirrupD}$  & $mm$\\
		Numero di bracci trasversali assegnato: & $n_{b,t}=\VAR{nbLegDirX}$  & $\ldots$\\
		\bottomrule
	\end{tabular}
\end{figure}

\begin{figure}[h!]
\centering
\begin{tabular}{p{4cm}ll} \toprule
	\multicolumn{3}{c}{\textbf{Armatura a flessione minima}} \\
    \midrule
	Descrizione & Valore & \mbox{u.d.m.} \\
    \midrule
    %%%%%%%%%%%%%%%%%%%%%%%%%%%%%%%%%%%%%%%%%%%%%%%%%%%%%%%%%%%%%%%%%%%%
	Altezza utile & $d=H-c-\cfrac{\phi_{lt}}{2}=\VAR{heightUtil}$  & $mm$\\
	Area utile & $A_u=b_{t,med} \cdot d=1000 \cdot d = \VAR{areaUtil}$  & $mm^2$\\
	Area minima criterio (1) & $A_{min}^{(1)}=0.26 \cdot \cfrac {f_{ctm}} {f_{yk}}\cdot A_u = \VAR{minimumRebarAreaCrit1}$ \marginnote{[4.1.45]} & $mm^2$\\
	Area minima criterio (2) & $A_{min}^{(2)}=0.0013 \cdot A_u= \VAR{minimumRebarAreaCrit2}$ \marginnote{\S 4.1.6.1.1} & $mm^2$\\
  	Area minima & $A_{min}=max(A_{min}^{(1)},A_{min}^{(2)})=\VAR{minimumRebarArea}$ & $mm^2$\\
	Area di acciaio disposta in zona corrente & $A_{disp}=\VAR{rebarAreaDisposed}$  & $mm^2$\\
	Numero di barre disposte in zona corrente & $n_{disp}=\VAR{rebarNumber}$  &  $\ldots$\\
    %%%%%%%%%%%%%%%%%%%%%%%%%%%%%%%%%%%%%%%%%%%%%%%%%%%%%%%%%%%%%%%%%%%%
    \bottomrule
\end{tabular}
\end{figure}


\begin{figure}[h!]
\centering
\begin{tabular}{p{4cm}ll} \toprule
	\multicolumn{3}{c}{\textbf{Armatura a taglio minima (spilli o staffe)}} \\
    \midrule
	Descrizione & Valore & \mbox{u.d.m.} \\
    \midrule
    %%%%%%%%%%%%%%%%%%%%%%%%%%%%%%%%%%%%%%%%%%%%%%%%%%%%%%%%%%%%%%%%%%%%
	Area minima per metro di elemento & $\cfrac{A_{sw,min}}{s} = 1.5 \cdot b_{t,min}=\VAR{minimumRebarAreaForElementLenght}$ \marginnote{\S 4.1.6.1.1} & $mm^2/m$\\
	Numero di bracci totale con diametro assegnato & $n_{b}=\VAR{minimumLegsForElementLenght}$  & $\ldots$\\
  	Passo massimo per disporre area minima (1) & $s_{max}^{(1)}=\cfrac{n_{b,t}}{n_{b}}\cdot 1000=\VAR{maxStepCrit1}$ & $mm$\\
    Passo massimo con altezza utile (2)        & $s_{max}^{(2)}=0.8 \cdot d =\VAR{maxStepCrit3}$ \marginnote{\S 4.1.6.1.1}  & $mm$\\
    Passo massimo assoluto di tre staffe al metro (3)        & $s_{max}^{(3)} = \VAR{maxStepCrit2}$ \marginnote{\S 4.1.6.1.1} & $mm$\\
	Passo massimo con barre compresse (4) & $s_{max}^{(4)}=15 \cdot \phi_{lc} = \VAR{maxStepCrit4}$ \marginnote{\S 4.1.6.1.1}  & $mm$\\
	Passo massimo in generale & $s_{max}=min(s_{max}^{(1)},s_{max}^{(2)},s_{max}^{(3)})=\VAR{stirrupStepMin}$  & $mm$\\
	Passo massimo con barre compresse & $s_{max}=min(s_{max}^{(1)},s_{max}^{(4)})=\VAR{stirrupCompStepMin}$  & $mm$\\
    %%%%%%%%%%%%%%%%%%%%%%%%%%%%%%%%%%%%%%%%%%%%%%%%%%%%%%%%%%%%%%%%%%%%
    \bottomrule
\end{tabular}
\end{figure}
